% arara: pdflatex: {synctex: yes, action: nonstopmode}
% arara: pdflatex
% arara: bibtex
% arara: pdflatex
% arara: pdflatex
% arara: nomencl
% arara: pdflatex: {synctex: yes, action: nonstopmode}

%\documentclass[a4crop]{ntnuthesis}
\documentclass{simthesis}
\usepackage[utf8]{inputenc}
\usepackage{subfiles}
\usepackage{amsmath} %equations and spacings
\usepackage[final]{pdfpages}
\usepackage{lipsum} % access to random text
%graphics
\usepackage{graphicx}
%\usepackage[demo]{graphicx}
%caption styles
\usepackage[small,bf]{caption}
\usepackage{caption}
\usepackage[labelformat=simple]{subcaption}
\renewcommand\thesubfigure{(\alph{subfigure})} % see subcaption doc
% the following format will be used to emulate the captions produced by fltpage
\DeclareCaptionLabelFormat{adja-page}{\hrulefill\\#1 #2 \emph{(previous page)}}

%<<byman
%\usepackage{xcolor}
\usepackage{color}
\usepackage{sidecap}
\definecolor{cffffff}{RGB}{255,255,255}
\definecolor{c2f528f}{RGB}{47,82,143}
\definecolor{c4472c4}{RGB}{68,114,196}

\def \globalscale {1.000000}

%byman>>

%<<byman
%% We want UK-english hyphenation patterns.byman
\usepackage[british]{babel}
%byman>>

%Remove Paragraph indenting
\setlength{\parindent}{0.0in}
\setlength{\parskip}{0.1in}

%<<byman
%% Use scalable, PostScript Type 1 versions of the Computer Modern fonts.
\usepackage{type1cm}
%% Replace the standard Computer Modern Typewriter font LaTeX uses
%% for monospace text with the PostScript font Adobe Courier.
\usepackage{courier}
\usepackage[T1]{fontenc}
\usepackage{ae,aecompl}
\usepackage{times}
%% Redefine the font used for the section headings to
%% Helvetica-Narrow Bold.
\usepackage{sectsty}
\allsectionsfont{\usefont{OT1}{phv}{bc}{n}\selectfont}
% for colour shades in tables
\usepackage{colortbl}
%to control the title formatting and spaces around the titles
\usepackage{pdfpages}
%%byman>>

%spaces under sections
\usepackage[compact]{titlesec}
\titlespacing{\section}{0pt}{*0}{*0}
\titlespacing{\subsection}{0pt}{*0}{*0}
\titlespacing{\subsubsection}{0pt}{*0}{*0}
%fonts
%\usepackage[T1]{fontenc} %gives light font
%\usepackage[light,math]{iwona}

%Abbreviations or Acronyms
%\usepackage[intoc]{nomencl}
\usepackage{nomencl}
\renewcommand{\nomname}{List of Abbreviations}
\makenomenclature

\hyphenation{significant}
\hyphenation{Optometry}
\hyphenation{diagnoses}

%\def\hyphenpenalty\10000
%\def\exhyphenpenalty\10000

\usepackage[a-1b]{pdfx}

%Bibliography
%\usepackage{url}
%\usepackage{natbib}
%\usepackage[]{natbib}

\usepackage[sectionbib,square,sort,comma,numbers]{natbib}
\usepackage{chapterbib}

%chapter biblio
%\usepackage[sectionbib]{chapterbib}
%\bibpunct[:]{(}{)}{;}{a}{}{,} %citation structure
%\bibpunct{(}{)}{,}{a}{}{;}
%\bibpunct{[}{]}{,}{a}{}{;}
%\bibpunct{(}{)}{;}{a}{}{,} % to follow the A&A style

\newcommand{\cmt}[1]{}%inline comment

%tabularx
\usepackage{tabularx}
\usepackage{tabulary}
\usepackage{tabu}
\usepackage{longtable} %table on several pages
\usepackage{booktabs}

%\newlength\mylen
%\setlength\mylen{2\tabcolsep}
%\addtolength\mylen{\arrayrulewidth}

%\usepackage{seqsplit}

\newcolumntype{s}{>{\hsize=.5\hsize}X}
% \newcolumntype{a}{\linewidth}{|*{3}{>{\hsize=0.4\hsize}X| >{\hsize=1.6\hsize}X|}}
\newcolumntype{D}{>{\small\RaggedRight\arraybackslash\hsize=0.4\hsize}X}
\newcolumntype{E}{>{\small\Centering\arraybackslash\hsize=1.6\hsize}X}
\setlength{\tabcolsep}{1.5pt}

\newcolumntype{L}[1]{>{\hsize=#1\hsize\raggedright\arraybackslash}X}%
\newcolumntype{R}[1]{>{\hsize=#1\hsize\raggedleft\arraybackslash}X}%
\newcolumntype{C}[2]{>{\hsize=#1\hsize\columncolor{#2}\centering\arraybackslash}X}%


\usepackage{enumerate}

%enumerate(control spacing)
\newenvironment{packed_enum}{
	\begin{enumerate}
		\vspace*{-1}
		\setlength{\itemsep}{1pt}
		      \setlength{\parskip}{0pt}
		      \setlength{\parsep}{0pt}
		      }{\end{enumerate}}

%<<byman
\usepackage{multirow}
\usepackage{dcolumn}
\newcolumntype{d}{D{.}{.}{-2}}
\newcommand{\tabincell}[1]{\begin{tabular}{c}#1\end{tabular}}
\newcommand{\tabincellt}[1]{\begin{tabular}{l}#1\end{tabular}}
\newcommand{\merge}[1]{\multicolumn{1}{c}{#1}}

\usepackage{array}
\newcommand{\PreserveBackslash}[1]{\let\temp=\\#1\let\\=\temp}
\newcolumntype{C}[1]{>{\PreserveBackslash\centering}p{#1}}
\newcolumntype{R}[1]{>{\PreserveBackslash\raggedleft}p{#1}}
\newcolumntype{L}[1]{>{\PreserveBackslash\raggedright}p{#1}}

% Used to create tables with rows/cols spanning over se
%% Use fancy chapter headers, with Jos Dingjan's modifications,
%% plus my own tweaks. This style is not part of teTeX,
%% so we are using a local (and renamed) copy.
%\usepackage[Lenny]{fncychapleo}
\usepackage{fncychap}

%% Nicely format and linebreak URLs in the bibliography (and elsewhere).
%\usepackage{url}
%% Define a new 'leo' style for the package that will use a smaller font.
\makeatletter
\def\url@leostyle{%
	\@ifundefined{selectfont}{\def\UrlFont{\sf}}{\def\UrlFont{\small\ttfamily}}}
\makeatother
%% Now actually use the newly defined style.
\urlstyle{leo}
%byman>>

%graphics
\usepackage{epstopdf} %support for eps.
\DeclareGraphicsExtensions{.eps,.ps,.pdf,.png,.jpg}
\usepackage{float} % figure placing [H]

%% landscape Option
\usepackage{lscape} %Left down
%usepackage{pdflscape} %Left up)

%\usepackage[draft]{todonotes}   % notes showed (JJUNJU)
%% Select what to do with command \comment:
% \newcommand{\comment}[1]{}  %comment not showed
\newcommand{\comment}[1]
{\par {\bfseries \color{red} #1 \par}} %comment showed

%% Section Numbering and TOC depth
\setcounter{secnumdepth}{2}
\setcounter{tocdepth}{3}

%dot2tex
%\usepackage[x11names, rgb]{xcolor}
\usepackage{etoolbox}

%\usepackage[nomain,acronym,xindy,toc]{glossaries} % nomain, if you define glossaries in a file, and you use \include{INP-00-glossary}

%\makeglossaries
%\usepackage[style=long,nonumberlist,toc,xindy,acronym,nomain]{glossaries}

%\makeglossaries
%\usepackage[xindy]{imakeidx}
%\makeindex

%% Drawing matplotlib plots

\usepackage{pgfplots}
\DeclareUnicodeCharacter{2212}{−}
\usepgfplotslibrary{groupplots,dateplot}
\usetikzlibrary{patterns,shapes.arrows}
\pgfplotsset{compat=newest}

%% Drawing flow charts
\usetikzlibrary{er,shapes,arrows,arrows.meta}
\usetikzlibrary{er,shapes,arrows,arrows.meta}

\tikzstyle{decision} = [diamond, draw, fill=blue!20,
    text width=4.5em, text badly centered, node distance=3cm, inner sep=0pt]
\tikzstyle{block} = [rectangle, draw, fill=blue!20,
    text width=5em, text centered, rounded corners, minimum height=4em]
\tikzstyle{line} = [draw, -latex']
\tikzstyle{cloud} = [draw, ellipse,fill=red!20, node distance=3cm,
    minimum height=2em]



% Define block styles
\tikzstyle{state}   = [ rounded rectangle,
						draw,
						text centered,
						minimum height = 3em,
						minimum width = 6em,
						inner sep = 5pt
						]
\tikzstyle{test}    = [ diamond,
						draw,
						shape aspect = 2,
						inner sep = 0pt,
						text width = 7em,
						text centered
						]
\tikzstyle{action}  = [ rectangle, draw,
						text width = 8em,
						inner sep = 5pt,
						minimum height = 5em
						]
\tikzstyle{data}    = [ trapezium,
						draw,
						trapezium left angle = 60,
						trapezium right angle = 120pt,
						minimum height = 6em,
						text width = 5em,
						inner xsep = 0pt
						]
\tikzstyle{line}    = [ draw, -triangle 45 ]

\usepackage{zref-savepos}
\providecommand*{\zsaveposy}{\zsavepos}% support older zref-savepos

\newdimen\flowheight



%\usepackage{pgf-pie}
\usetikzlibrary{arrows.meta}
\tikzset{%
>={Latex[width=2mm,length=2mm]},
%% Specifications for style of nodes:
base/.style = {rectangle, rounded corners, draw=black,
		minimum width=4cm, minimum height=1cm,
		text centered, font=\sffamily},
context/.style = {base, fill=blue!30},
futureWork/.style = {base, fill=green!30},
process/.style = {base, minimum width=2.5cm, fill=orange!15, font=\ttfamily},
}

\usepackage{smartdiagram}
%\usepackage{pygmentex}
\usepackage{pythonhighlight}
%\usepackage{bookcover}
% \usepackage{svg}

%\usepackage{etoc}

%% To generate list of symbols
\newcommand{\addsymbol}[3]{%
	\symboldisplay{#1}{#2}\\%
	\setelem{#3}{#1}
}
\newcommand{\symboldisplay}[2]{%
	$#1$ \parbox{5in}{\dotfill #2}%
	%$#1$ \parbox{5in}{ #2}%
}
%\def\setelem#1{\expandafter\def\csname myarray(#1)\endcsname}
\def\setelem#1{\expandafter\gdef\csname myarray(#1)\endcsname}
\def\dispsymbol#1{\csname myarray(#1)\endcsname}

%% Chapter numbers
\newcommand{\chap}[2]{
	\setcounter{chapter}{#1}
	\setcounter{section}{0}
	\refstepcounter{chapter}{#1}
	\chapter*{#2}
	\addcontentsline{toc}{chapter}{#2}
}

%% Chapter subtitles: `\Chapter{My chapter}{Nice subtitle}`
\newcommand\Chapter[2]{\chapter
	 [#1\hfil\hbox{}\protect\linebreak{\itshape#2}]%
	 {#1\\[2ex]\Large\itshape#2}%
}
\newcommand\ChapterNoNo[2]{\chapter*
	 [#1\hfil\hbox{}\protect\linebreak{\itshape#2}]%
	 {#1\\[2ex]\Large\itshape#2}%
}


%Foo bar\acrshort{lvm} can haz\acrshort{ddye} bzr.

%<<byman
%% Configuration of the header strings for the frontmatter pages.
% \fancyhead[RO]{{\footnotesize\rightmark}\hspace{2em}\thepage}
% \setcounter{tocdepth}{2}
% \fancyhead[LE]{\thepage\hspace{1em}\footnotesize{\leftmark}}
% \fancyhead[RE,LO]{}
% \fancyhead[RO]{{\footnotesize\rightmark}\hspace{2em}\thepage}
%byman>>

%\fancyhf{}
%\fancyhead[LE,RO]{\thepage}
%\fancyfoot[RE]{\textit{ \nouppercase{\leftmark}} }
%\fancyfoot[LO]{\textit{ \nouppercase{\rightmark}} }

\fancyhf{}
\fancyfoot[C]{}

\iffalse
	\listoftables
	\addcontentsline{toc}{chapter}{List of Tables}
	\cleardoublepage

	\listoffigures
	\addcontentsline{toc}{chapter}{List of Figures}
	\cleardoublepage

	%% Abbreviations

	\newpage
	\printnomenclature
	\cleardoublepage

	%% Symbols
	\newpage
	\chapter*{List of Symbols\hfill}
	\addtocontents{toc}{\protect\thispagestyle{empty}}
	\pagenumbering{gobble}
	\begin{flushleft}
		%	\input{symbols}
	\end{flushleft}
\fi

\robustify\textemdash

\usepackage{hyperref}
%\hypersetup{colorlinks=true,citecolor=blue}
%\hypersetup{colorlinks=true,citecolor=blue,linkcolor=blue,urlcolor=blue}
%\hypersetup{colorlinks=false}
\hypersetup{colorlinks=true,citecolor=black,linkcolor=black,urlcolor=black}

%% To change box color around the links and citations, you have these other options :
%\hypersetup{citebordercolor=Violet,filebordercolor=Red,linkbordercolor=Blue}
%\hypersetup{citebordercolor=Red,filebordercolor=White,linkbordercolor=White}

\usepackage{bookmark}
