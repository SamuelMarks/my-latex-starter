% the sample slide is created with 16:9 aspect ratio
\documentclass[aspectratio=169,xcolor={dvipsnames}]{beamer}

\definecolor{foobarblue}{RGB}{0,153,255}
\definecolor{foobaryellow}{RGB}{234,187,0}

\usepackage{hyperref}
\usepackage{pgfgantt}
\usetikzlibrary{shapes.geometric}

\newcommand{\Csharp}{%
  {\settoheight{\dimen0}{C}C\kern-.05em \resizebox{!}{\dimen0}{\raisebox{\depth}{\#}}}}

\DeclareUnicodeCharacter{2026}{\textellipsis}

\usepackage[acronym,xindy,toc,nomain]{glossaries-extra} \makeglossaries

\newacronym{OS}{OS}{Operating System}
\newacronym{ML}{ML}{Machine Learning}
\newacronym{FFI}{FFI}{Foreign Function Interfaces}
\newacronym{CDD}{CDD}{Compiler Driven Development}
\newacronym{PT}{PT}{Personal Trainer}
\newacronym{SDK}{SDK}{Software Development Kit}
\newacronym{CLI}{CLI}{Command Line Interface}
\newacronym{ORM}{ORM}{Object Relational Mapper}

% remove the options if you do not want to have them
\usetheme[
	% background=images/background.jpg, % you can add your own background image
	logo=images/unsw-portrait.png,
	sidelogo=images/unsw-landscape.png,
]{unsw}
% uncomment to show notes. Works very nicely with dspdfviewer. You get something similar to PPT's presenter view.
%\usepackage{pgfpages}
%\setbeameroption{show notes on second screen}

% information for the title page
\author{Samuel Marks, PhD}
\title{Multi\small{-}\LARGE{}ML cross\small{-}\LARGE{}platform meta programming at scale using compiler tech and C}
\subtitle{PhD annual review, August 2022}
\institute{Computer Science and Eng}
\date{\today}

% - Gantt Chart
% - Be less concise
% - Reference NAS
% - 

\begin{document}
	% use plain option to remove the page number from the title slide
	\begin{frame}[plain]
		\titlepage
	\end{frame}

	\begin{frame}{Outline}
		\tableofcontents
	    % You might wish to add the option [pausesections]
	\end{frame}

	\begin{frame}{Background}
		In 1968 the NATO SCIENCE COMMITTEE had a conference\textemdash{}and produced a report\textemdash{}on `software engineering'.\cite{naurSoftwareEngineering1968}. This was a landmark event, spruning the formation of `software engineering' as a discipline; culminating in a general appreciation of a crisis in programming.\cite{macroCraftSoftwareEngineering1987}\\[2.5mm]

		Focussed on in my research is the proceeding\textemdash{}from that NATO conference\textemdash{}on software components \cite{mcilroyMassProducedSoftwareComponents1968}, i.e., mass development of mass customisable software.\\[2.5mm]

		Related foci in my research are: problems of deployment at scale; keeping up-to-date with the massive amount of new \glsentrylong{ML}; and pulling all these together to facilitate \textit{scalable rapid software engineering}.
	\end{frame}


	\begin{frame}{bonâ fides}
		\textbf{Work (since 2009)}
		\begin{enumerate}
			\item[0.] Tech support;
			\item Software engineer;
			\item Senior technology specialist;
			\item Senior software engineer
			\item Lead software engineer
			\item Head/director (running my own engineering consultancy)
		\end{enumerate}		

		\textbf{Education}
		\begin{enumerate}
			\item[0.] 2014: Bachelor of Science (wherein I took most all required to get all Computer Science, Information Technology, and Business Information Systems majors) from Macquarie University.
			\item 2021: Doctor of Philosophy (PhD) from the University of Sydney.
		\end{enumerate}
	\end{frame}

	\begin{frame}{Problem (0/1)}
		My experience, verified by speaking at dozens and attended{\raise.17ex\hbox{$\scriptstyle\sim$}}100 industry technology events, and further confirmed from the literature, exposed major problems in software-engineering. The ones I am focussing on in this PhD are:
		\begin{itemize}
			\item Native development benefits from targeting iOS, macOS, Windows, Linux, Android, and Linux.\\
			\(\hdots\)but this greatly reduces development agility and quality metrics like consistency and test \& doc coverage.
			\item \glsentrylong{ML} field changes too quickly for anyone to keep up-to-date with. Meaning one cannot claim superiority of their solution across industry/academia.
		\end{itemize}
	\end{frame}

	\begin{frame}{Problem (1/1)}
		\begin{itemize}
			\item The same stack cannot be deployed for: embedded (e.g., whole stack on smarthpone); peer-to-peer; and client/server architectures.
			\item Deploying to different clouds requires a different knowledge set.
		\end{itemize}
	    
		In summary, developing quality, complex and multi-tier systems requires:
		\begin{itemize}
			\item a team;
			\item increasingly specialised knowledge;
			\item lots of time; and
			\item an expansive budget.
		\end{itemize}
	\end{frame}

	\begin{frame}{Solution}
		Write compilers.
	\end{frame}
		
	\begin{frame}[t, allowframebreaks]
		\frametitle{References}
		\bibliographystyle{amsalpha}
		\bibliography{Thesis}
	\end{frame}
\end{document}
